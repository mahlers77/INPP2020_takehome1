\documentclass[a4paper,11pt]{report}
\usepackage[margin=2.cm]{geometry}

\usepackage{amssymb}
\usepackage{amsmath}
\usepackage{amsfonts}
\usepackage{graphicx}

\usepackage{times}
%\usepackage{txfonts}
\usepackage{bm}

\usepackage{color}

\linespread{1}
\setlength{\parskip}{1.0ex}
\setlength{\parindent}{0.0ex}

\begin{document}

\begin{center}
{\large \textsc{Introduction to Nuclear and Particle Physics}} \\[0.5cm]
\Large{Take-Home Exam I\\
Due: Monday $14^{th}$ December 2020 at 13:00}
\end{center}

{\large \textsc{Problem 1 - Kaon Production Threshold} - XY points}

\noindent Consider a $\pi^-$ beam impinging on a liquid hydrogen target. Find the threshold energy of $K^-$ production assuming strong interactions.
\vspace{0.5cm}

{\large \textsc{Problem 2 - Conserved Quantum Numbers} - XY points}

\noindent List all types of interactions -- strong, electromagnetic, and weak -- that conserve the following individual quantum numbers:

\noindent 2.1) $I$ : isospin\\
\noindent 2.2) $I_3$ : third component of isospin\\
\noindent 2.3) $\widetilde{B}$ : beauty (or bottom-ness)\\
\noindent 2.4) $L_\tau$ : tau lepton number\\
\noindent 2.5) $B$ : baryon number\\
\noindent 2.6) $L$ : total lepton number\\
\noindent 2.7) $J$ : total angular momentum\\
\noindent 2.8) $J_3$ : third component of total angular momentum\\
\noindent 2.9) $S$ : strangeness\\
\noindent 2.10) $C$ : charm
\vspace{0.5cm}

{\large \textsc{Problem 3 - Topic} - XY points}

{\large \textsc{Problem 4 - Topic} - XY points}

{\large \textsc{Problem 5 - Topic} - XY points}

{\large \textsc{Problem 6 - Topic} - XY points}
\end{document}